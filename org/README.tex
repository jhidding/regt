% Created 2018-04-10 di 10:37
% Intended LaTeX compiler: pdflatex
\documentclass[11pt]{article}
\usepackage[utf8]{inputenc}
\usepackage[T1]{fontenc}
\usepackage{graphicx}
\usepackage{grffile}
\usepackage{longtable}
\usepackage{wrapfig}
\usepackage{rotating}
\usepackage[normalem]{ulem}
\usepackage{amsmath}
\usepackage{textcomp}
\usepackage{amssymb}
\usepackage{capt-of}
\usepackage{hyperref}
\author{Johannes Hidding}
\date{\today}
\title{}
\hypersetup{
 pdfauthor={Johannes Hidding},
 pdftitle={},
 pdfkeywords={},
 pdfsubject={},
 pdfcreator={Emacs 25.2.2 (Org mode 9.1.9)}, 
 pdflang={English}}
\begin{document}

\tableofcontents

\section{Regular triangulations for Cosmology}
\label{sec:orgbf10868}

Regular triangulations are a very useful tool compute the \textbf{geometric
adhesion model}.  This code solves Burgers' equation

\[\frac{\partial u}{\partial t} + (u \cdot \grad)u = \nu \grad^2 u.\]

This equation is solved using geometric methods available in the
Computational Geometry Algorithms Library (CGAL).

\subsection{Features}
\label{sec:orgb3c7d53}
\begin{itemize}
\item generate initial conditions
\item create glass files
\item compute Zeldovich displacements
\item compute the adhesion model
\item work in 2D or 3D
\item output in PLY
\end{itemize}

\subsubsection{future}
\label{sec:org48fa9d4}
\begin{enumerate}
\item (merge from git:jhidding/adhesion-example repo)
\label{sec:orgf6612ee}
\begin{itemize}
\item OBJ triangle output (splits polygons into triangles, but renders better)
\item Lloyd iteration glasses
\end{itemize}
\item Todo
\label{sec:org3579565}
\begin{itemize}
\item periodic triangulations
\item use HDF5
\item increase test coverage
\end{itemize}
\end{enumerate}

\subsection{Output}
\label{sec:org06f27f0}
The output of the 3D adhesion code is in PLY format. PLY is a very liberal specification that
supports storing polygons of any number. Filaments and wall structures are stored in different
files (clusters are stored separately). We associate a density with each element in the PLY file.
This is within specification but not \emph{canon}, so some applications claiming to support PLY
(like paraview) may crash trying to load these files.

\subsection{Build}
\label{sec:orgb7951f2}
\subsubsection{Prerequisites}
\label{sec:orgf108ac3}
\begin{itemize}
\item CGAL >= 4.11
\item C++17 compatible compiler (GCC >= 7, LLVM/clang >= 4)
\item Meson / ninja build system
\item FFTW3
\item GNU Scientific Library (GSL)
\item (for unit testing only) GTest
\end{itemize}

\subsubsection{Building}
\label{sec:orgf598aa9}
In the project folder, run \texttt{meson} and subsequently \texttt{ninja}.
\begin{verbatim}
$ meson build --buildtype release
...
$ cd build
build$ ninja
\end{verbatim}

To run unit tests (currently only tests PLY output code)

\begin{verbatim}
build $ ninja test
\end{verbatim}
\end{document}
